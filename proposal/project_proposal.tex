\documentclass[letterpaper, 11pt]{article}
\usepackage[left=2cm, top=2.5cm, right=2cm, bottom=2cm]{geometry}

\usepackage[colorlinks = true,
linkcolor = blue,
urlcolor  = blue,
citecolor = blue,
anchorcolor = blue]{hyperref}

\usepackage[dvipsnames]{xcolor}

\newcommand\bb[1]{\mbox{\em #1}}
\def\baselinestretch{1.05}

\newcommand{\hsp}{\hspace*{\parindent}}
\definecolor{gray}{rgb}{0.4,0.4,0.4}

\usepackage{siunitx}
\sisetup{range-phrase={\text{--}},range-units=single}

\usepackage{multibib}
\newcites{main}{References}

\linespread{1.0}

\hyphenation{ATLAS}

\usepackage{fancyhdr}
\usepackage{pagecounting}
\fancyhf{}
\renewcommand{\headrulewidth}{0pt}
\renewcommand{\footrulewidth}{0pt}

\newcommand{\institute}{UW--Madison}
\newcommand{\fullinstitute}{University of Wisconsin--Madison}
\newcommand{\department}{Physics}
\newcommand{\lhcexperiment}{ATLAS}

\newcommand{\pyhf}{\texttt{pyhf}}

\lhead{
  \textcolor{gray}{\textit{Matthew Feickert}}\\
  \href{mailto:matthew.feickert@cern.ch}{matthew.feickert@cern.ch}
}
\rhead{\textcolor{gray}{\href{https://urssi.us/blog/2024/11/22/call-for-proposals-for-the-urssi-early-career-fellowship-program/}{URSSI Fellowship Winter 2025}\\%
\thepage/\totalpages{}}}
\chead{\Large\scshape\color[HTML]{CF000F} Project Proposal}

\begin{document}
\pagestyle{fancy}

% * [ ] A concise research proposal (e.g. 2-3 pages)

% * Make sure to include short (on the road to tenure), medium (10-15 years), and long term ideas (where is career going post LHC) in the research statement

% Successful candidates for the second position on “scientific software infrastructure” will seek to advance the **reliable and efficient development and application of artificial intelligence/machine learning methods through the development of scalable, extensible, and interoperable software, including platforms that support computational reproducibility and open science**.
% Scientific domain questions of interest include the data-driven discovery of new materials to address climate change based on AI/ML and high-dimensional sparse statistics.
% We will consider creative and energetic candidates who show promise and/or accomplishment in research, teaching, and mentoring.
% The Department of Statistics offers a PhD program, a professional master’s program, and undergraduate majors in Statistics and in Data Science (the latter jointly with EECS).
% In particular, the Statistics and Data Science majors provide opportunities to develop domain emphasis courses related to scientific computing and machine learning in chemistry, physics, and materials science.
% The person offered this position would have the opportunity to engage in teaching, course development, and mentoring for each of these programs.
% They would join a diverse group of colleagues in the Departments of Statistics and EECS, with world-renowned expertise in both AI/ML and open-source scientific computing, and a sustained tradition of collaboration in data science with disciplines spanning the natural and social sciences.
% They would have the opportunity to interact and collaborate with colleagues in the Departments of Chemistry, Materials Science and Engineering, and Physics, among others, as well as a possible future interdisciplinary department in CDSS, BIDMaP, and the nearby Lawrence Berkeley National Laboratory.

% * this is mainly written to a HEP audience, but the opening is for any experimental physics. I would add a little bit at the beginning to contextualize your work in HEP for a non-HEP audience.
%    - this also applies to the cover letter. Re-read your opening paragraph with the idea that the audience includes non HEP people.
% * You don’t say much about AI. One opportunity is in section 3 when you talk about SBI. You might make it explicit that the SBI revolution is powered by AI/ML and also maybe add a few sentences for those that don’t know what SBI is.
%    - you might consider putting AI into the section title for section 3
% * in section 4 you might connect differentiable programming and automatic differentiation to AI. You may know about the quote from Yann LeCun “deep learning is dead, long live differentiable programming”

\section{Project Goals and Objectives}

State the track that your project is most closely aligned with; Provide a brief overview of the project including related research, and previous efforts; and, Provide clear, measurable goals that will be accomplished within 6 months. Include specific objectives that contribute to the broader URSSI mission and research track.

\section{Expected Impact on Scientific Software Community}

Direct and indirect benefits to the research software community. How will your work improve scientific software development practices?

\section{Implementation Plan}

Month-by-month breakdown of activities, methods, and approaches. Include specific milestones and check-in points.

\section{Community Engagement Strategy}

How will you involve and benefit the broader scientific software community? Include specific outreach and collaboration plans.

\section{Evaluation Metrics}

Concrete ways to measure project success. Include both quantitative and qualitative metrics.

\section{Timeline and Deliverables}

Specific six-month timeline with monthly milestones and concrete deliverables. Example structure:

\begin{itemize}
  \item Month 1: Project setup and initial research
  \item Month 2-3: Development/Investigation phase
  \item Month 4-5: Testing/Validation/Analysis
  \item Month 6: Documentation and dissemination
\end{itemize}

\section{Summary}

If needed

%
\vspace*{-0.25cm}
\begin{footnotesize}
%
 \bibliographystylemain{atlasBibStyleWithTitle}
 \bibliographymain{main}
 %\nocitemain{*}
%
\end{footnotesize}
\end{document}
