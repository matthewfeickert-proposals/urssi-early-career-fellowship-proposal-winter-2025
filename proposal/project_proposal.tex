\documentclass[letterpaper, 11pt]{article}
% \usepackage[left=2cm, top=2.5cm, right=2cm, bottom=2cm]{geometry}
\usepackage[left=1.5cm, top=2.5cm, right=1.5cm, bottom=1.5cm]{geometry}

\usepackage[colorlinks = true,
linkcolor = blue,
urlcolor  = blue,
citecolor = blue,
anchorcolor = blue]{hyperref}

\usepackage[dvipsnames]{xcolor}

\newcommand\bb[1]{\mbox{\em #1}}
\def\baselinestretch{1.05}

\newcommand{\hsp}{\hspace*{\parindent}}
\definecolor{gray}{rgb}{0.4,0.4,0.4}

\usepackage{siunitx}
\sisetup{range-phrase={\text{--}},range-units=single}

\linespread{1.0}

\hyphenation{ATLAS}

\usepackage{fancyhdr}
\usepackage{pagecounting}
\usepackage{enumitem}

\fancyhf{}
\renewcommand{\headrulewidth}{0pt}
\renewcommand{\footrulewidth}{0pt}

\newcommand{\institute}{UW--Madison}
\newcommand{\fullinstitute}{University of Wisconsin--Madison}
\newcommand{\department}{Physics}
\newcommand{\lhcexperiment}{ATLAS}
\newcommand{\fullprogram}{US Research Software Sustainability Institute}
\newcommand{\program}{URSSI}

\newcommand{\pyhf}{\texttt{pyhf}}
\newcommand{\pixi}{\texttt{pixi}}

\newcommand{\milestone}{\textbf{\textrm{M}}}
\newcommand{\deliverable}{\textbf{\textrm{D}}}

\lhead{
  \textcolor{gray}{\textit{Matthew Feickert}}\\
  \href{mailto:matthew.feickert@cern.ch}{matthew.feickert@cern.ch}
}
\rhead{\textcolor{gray}{\href{https://urssi.us/blog/2024/11/22/call-for-proposals-for-the-urssi-early-career-fellowship-program/}{URSSI Fellowship Winter 2025}\\%
\thepage/\totalpages{}}}
\chead{\Large\scshape\color[HTML]{CF000F} Project Proposal}

\begin{document}
\pagestyle{fancy}

\section{Project Goals and Objectives}

% * State the track that your project is most closely aligned with.

% * Provide a brief overview of the project including related research, and previous efforts.
% * Provide clear, measurable goals that will be accomplished within 6 months.
% * Include specific objectives that contribute to the broader URSSI mission and research track.

A critical component of research software sustainability is the reproducibility of the software and computing environments software operates and ``lives'' in.
Providing software as a ``package'' --- a standardized distribution of all source or binary components of the software required for use along with identifying metadata --- goes a long way to improved reproducibility of software libraries.
However, more researchers are consumers of libraries than developers of them, but still need reproducible computing environments for research software applications that may to be run across multiple computing platforms --- e.g. scientific analyses, visualization tools, data transformation pipelines, and artificial intelligence (AI) and machine learning (ML) applications on hardware accelerator platforms (e.g. GPUs).
While workflow engines and Linux containers offer a gold standard for scientific computing reproducibility, they require additional layers of training and software engineering knowledge.
Modern open source multi-platform environment management tools, e.g. \pixi{}~\cite{Arts_pixi}, provide automatic multi-platform hash-level lock file support for all dependencies --- down to the compiler level --- of software on public package indexes (e.g. PyPI~\cite{PyPI_website} and conda-forge~\cite{conda-forge_community}) while still providing a high level interface well suited for researchers.
We are now at a point where well supported, robust technological solutions exist, even for applications with highly complex software environments.
What is currently lacking is the education and training by the broader scientific software community to adopt these technologies and build community standards of practice around them, as well as an understanding of what are the most actionably useful features of adopting computational reproducibility tools.

\subsection{Project Focus and Goals}

As a \fullprogram{} (\program{}) Fellow, I will develop an open source course (website) on creating reproducible software environments for scientific applications including AI/ML applications, which require specialized hardware accelerator support.
The course will cover using \pixi{} to create and use reproducible software environments for Python applications using version controlled research software and software distributed on public package indexes, but will be scoped broadly to cover applications for \texttt{C}, \texttt{C++}, and Fortran libraries on conda-forge as well.
There will be a particular focus on using these tools for creating reproducible environments (even including all CUDA~\cite{CUDA_paper} dependencies) for AI/ML applications at scale, which is becoming a more relevant issue as AI/ML becomes a standard part of modern science.
I will also create a workshop targeting the broader scientific software community to teach the content of the course to researchers from multiple domains.
The workshop will be incubated at the \fullinstitute{} (\institute{}) Data Science Hub (DSH)~\cite{data_science_hub} with additional support from the Data Science Institute~\cite{uwmadison_dsi}, and will solicit a national call for applications for workshop participants, whose travel and lodging will be supported by the Fellow award funds as covered in the included proposed budget.
A version of the course will also be submitted as a tutorial proposal at the 2025 SciPy conference~\cite{scipy-2025}, and will be recommended as an optional module for future URSSI workshops and schools.
%
% \textbf{TODO: Mention that pixi is chosen given its features, open source, and that the pixi development team has expressed their support and willingness to collaborate on the proposed project.}

\subsection{Project Objectives and Deliverables}
This project will deliver permissively licensed open source educational material --- intended for community reuse --- and executable examples for real scientific software applications with a focus on AI/ML.
The material will be created with the intent to contribute it as a lesson module to The Carpentries Incubator~\cite{carpentries_incubator_proposals} and, once peer-reviewed, The Carpentries Lab~\cite{carpentries_lab} curriculum, where it will be maintained as an open source community resource.
It will also result in the training of a cohort of early career researchers across multiple scientific domains on best practices for reproducible scientific applications and will additionally identify areas of high complexity and significant difficulty in their current workflows that can either be alleviated or addressed through further technological development.
It will additionally serve as a testing ground for \pixi{} to handle real world complex scientific workflows and identify edge cases that will be addressed upstream in the \pixi{} GitHub repository in collaboration with the \pixi{} developers.
Each workshop will have a pre and post-workshop questionnaire that will be used to not only evaluate the clarity of the content and the efficiency at conveying the material, but also the level of information that the participants learned and found \emph{actionably useful} in their own research post-workshop.
This actionable information will be highlighted as a primary focus of the revised course material.

This project proposal is best aligned with the \textbf{Scientific Software Sustainability} track as it provides actionable methods that can be adopted by the broader scientific software community for better software sustainability and reproducibility.
It also has significant overlap with the \textbf{Software Education Research} track, in that project products could be used in future \program{} schools, though the focus is on adoption of technologies and best practices rather than pedagogical research.
%
% \textbf{TODO: Clarify how this is different than MLFlow frozen environments.}

\section{Expected Impact on Scientific Software Community}

% Direct and indirect benefits to the research software community.
% How will your work improve scientific software development practices?

Reproducible software environments are critical not only for reproducibility of results, but, arguably more importantly, also for practically sharing complex scientific environments with many dependencies for researchers to work collaboratively.
The open source website, course, and workshop series together will teach researchers how to make their software reusable and their exact software environments easily and immediately shareable.
It is without exaggeration that today researchers can spend hours building bespoke software environments for their work that have dependency trees so complex they require great effort to manually rebuild even on the same machines.
Providing researchers with a best practice workflow for lock-file-based reproducible software environments, and the tools to immediately implement them across multiple platforms, will scale the human time savings across science significantly as researchers adopt these methods across their research groups.
It will also increase computational savings by avoiding running of scientific workflows with improperly defined environments that could alter results, fail to reproduce key findings, or result in errors and crashes.
As GPU based workflows are notoriously difficult for sharing their exact environment specifications, this will provide a huge reproducibility benefit.
To ensure the highest technical quality of the course, the \pixi{} development team have agreed to provide a technical review of the course content.
Additionally, I will share the course widely and encourage other researchers to adopt it in their own research and teaching.

% Most researchers do not receive a formal education in software development and engineering and the associated best practices.
% However, these skills are vital to tackle the complex scientific questions of today.
Through the course and workshops I will be able to train early career researchers across multiple scientific domains on how to improve the maintainability of their software, reproducibility of their work, and ultimately the quality of their science.
The impact of this investment in their technical education will grow over time as they apply what they learned in their research and spread their knowledge in new collaborations.
% All the course and workshop content, and the website source code, will be open source under a permissive license, making it easy for other researchers to use the course material in their own lectures and workshops.
% Access to high quality scientific computing educational resources is an equity issue and open sourcing that knowledge makes it more accessible. The course will also not be static, but will be updated as new improvements and recommendations are made available.
% I will encourage sharing of the material and by having the workshops target a wide range of scientific domains I will ensure that this knowledge spreads and helps multiple scientific communities develop better workflows and practices around reproducible and sustainable software development.

\section{Implementation Plan}

% Month-by-month breakdown of activities, methods, and approaches.
% Include specific milestones and check-in points.

% The timeline and deliverables are what they stated earlier. In contrast the implementation plan is the “how” of how are you going to accomplish that timeline. What are the resources you need to get this work done? You will explain that with the milestones from the timeline but you are contextualizing how you plan to get the work done (e.g. who is on your team? how will your team meet/work?)
\textbf{Month 1} will focus on the majority of the course development and will instigate the process of collaborating with The Carpentries~\cite{the_carpentries_org}.
This will include reviews of best practices, and methodological challenges, of sustainable scientific software development so that in addition to technological approaches, techniques to addresses challenges can be addressed in the course.
\textbf{Month 2} will focus on contracting the \pixi{} development team for technical review of the course content, and the DSH for course structure review and assistant instructors.
It will also include the \emph{milestone} of designing the pre and post-workshop questionnaires to ensure they are designed to measure the amount of information learned about reproducible software environment workflows and determine what information provided the biggest actionable use to participants.
\textbf{Month 3} will provide a check-in point to evaluate any problems with the course development and the readiness for beginning the incubated workshops.
If there are problems, this will allow for corrections to be made and adjustments to the workshop schedules, if needed.
It will also focus on the first DSH incubated workshop \emph{deliverable} at \institute{} and opening the call for applications for the national workshop.
\textbf{Month 4} will focus on the \emph{deliverable} of a second revised iteration of the workshop with the DSH facilitators and the \emph{milestone} of selecting applicants for the national workshop and offering workshop travel and lodging stipends to non-local participants.
The final check-in point for establishing readiness for the national workshop will follow the second DSH incubated workshop.
\textbf{Month 5} will focus on the \emph{deliverable} of the national workshop at the DSH and the \emph{milestone} of assessing learned information actionability.
\textbf{Month 6} will focus on the \emph{deliverables} of teaching the course as a tutorial at SciPy 2025, submitting the course to The Carpentries Lab~\cite{carpentries_lab} for peer-review, and publishing the anonymized post-workshop feedback analysis results as an online interactive report generated with MyST Markdown~\cite{executable_books_community}.

\section{Community Engagement Strategy}

% How will you involve and benefit the broader scientific software community?
% Include specific outreach and collaboration plans.

While incubating the course at the DSH it will be submitted as a proposed lesson module on sustainable scientific software development for AI/ML applications to The Carpentries Incubator with the goal of having The Carpentries peer-review the course and accept it into The Carpentries Lab curriculum.
The Carpentries curriculum is curated, maintained, and taught globally to members of the scientific software community.
The DSH has experience working with The Carpentries and successfully contributing curriculum materials~\cite{backhaus_2024_14360351}.
% Additionally, the DSH incubated workshops, the national workshop at the DSH, and the SciPy 2025 tutorial will allow for a large number of early career researchers to be trained in sustainable best practices and learn on real scientific problems that they have in their own research programs.

\section{Evaluation Metrics}

% Concrete ways to measure project success.
% Include both quantitative and qualitative metrics.

% The project focuses on \textbf{Scientific Software Sustainability} by educating early career researchers on best practices of reproducible scientific analysis, with a focus on AI/ML applications.

Metrics for project success will include:
\setlist{nolistsep}
\begin{itemize}[noitemsep]
  \item The total number of participants across all the workshops. (Target: 50 participants)
  \item The percentage of participants who were able to successfully reproduce their own scientific and AI/ML research workflows using the information they learned by the end of the workshop. (Target: $90\%$)
  \item The acceptance of the course into The Carpentries Incubator and The Carpentries Lab curriculum.
\end{itemize}

\section{Timeline and Deliverables}

% Specific six-month timeline with monthly milestones and concrete deliverables.
% Example structure:

% \begin{itemize}
%   \item Month 1: Project setup and initial research
%   \item Month 2-3: Development/Investigation phase
%   \item Month 4-5: Testing/Validation/Analysis
%   \item Month 6: Documentation and dissemination
% \end{itemize}

Milestones and deliverables are noted by \milestone{} and \deliverable{}, respectively.

\begin{itemize}
  \item \textbf{Month 1} (February 2025): Project setup
    \begin{itemize}
      \item Develop a first draft of the course.
      \item \milestone{} Submit tutorial proposal for the course to SciPy 2025.
      \item \milestone{} Begin the process of contributing the course to the The Carpentries by creating a proposal GitHub issue with The Carpentries Incubator~\cite{carpentries_incubator_proposals}.
    \end{itemize}
  \item \textbf{Month 2} (March 2025): Iteration and revision
    \begin{itemize}
      \item Finalize the first iteration of the course content.
      \item Contract the DSH staff to review and provide feedback on the pedagogical structure of the course.
      \item Contract \pixi{} developers for a technical review of the course content.
      \item \milestone{} Create the pre-workshop and post-workshop questionnaires in consultation with DSH.
    \end{itemize}
  \item \textbf{Month 3} (April 2025): Incubation and testing
    \begin{itemize}
      \item \deliverable{} Run the first iteration of the DSH incubated workshop for \institute{} university students.
      \item \milestone{} Open a call for applications for the national workshop.
    \end{itemize}
  \item \textbf{Month 4} (May 2025): Incubation and revision
    \begin{itemize}
      \item \deliverable{} Run the second iteration of the DSH incubated workshop for \institute{} early career researchers.
      \item \milestone{} Select applicants for acceptance to attend the in-person national workshop.
      \item Revise the course content based on post-workshop questionnaire feedback.
    \end{itemize}
  \item \textbf{Month 5} (June 2025): Implementation of production course
    \begin{itemize}
      \item \deliverable{} Run the national workshop at the DSH.
      \item \milestone{} Evaluate the amount of information learned by participants and the actionability of that information in their own research.
      \item Revise the course content based on post-workshop questionnaire feedback.
    \end{itemize}
  \item \textbf{Month 6} (July 2025): Dissemination with broader community and documentation
    \begin{itemize}
      \item \deliverable{} Teach the course as a tutorial at SciPy 2025 international conference (pending acceptance).
      \item \deliverable{} Submit the course to be peer-reviewed for approval to be included in The Carpentries Lab.
      \item \deliverable{} Publication of the anonymized post-workshop questionnaire results as an interactive report generated with MyST Markdown.
    \end{itemize}
\end{itemize}

\clearpage
%
\vspace*{-0.25cm}
\begin{footnotesize}
%
 \bibliographystyle{atlasBibStyleWithTitle}
 \bibliography{main}
%
\end{footnotesize}
\end{document}
