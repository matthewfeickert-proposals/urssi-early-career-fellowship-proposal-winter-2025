\documentclass[letterpaper, 11pt]{article}
\usepackage[left=2cm, top=2.5cm, right=2cm, bottom=2cm]{geometry}

\usepackage[colorlinks = true,
linkcolor = blue,
urlcolor  = blue,
citecolor = blue,
anchorcolor = blue]{hyperref}

\usepackage[dvipsnames]{xcolor}

\newcommand\bb[1]{\mbox{\em #1}}
\def\baselinestretch{1.05}

\newcommand{\hsp}{\hspace*{\parindent}}
\definecolor{gray}{rgb}{0.4,0.4,0.4}

\usepackage{siunitx}
\sisetup{range-phrase={\text{--}},range-units=single}

\linespread{1.0}

\hyphenation{ATLAS}

\usepackage{fancyhdr}
\usepackage{pagecounting}
\fancyhf{}
\renewcommand{\headrulewidth}{0pt}
\renewcommand{\footrulewidth}{0pt}

\newcommand{\institute}{UW--Madison}
\newcommand{\fullinstitute}{University of Wisconsin--Madison}
\newcommand{\department}{Physics}
\newcommand{\lhcexperiment}{ATLAS}
\newcommand{\fullprogram}{US Research Software Sustainability Institute}
\newcommand{\program}{URSSI}

\newcommand{\pyhf}{\texttt{pyhf}}
\newcommand{\pixi}{\texttt{pixi}}

\newcommand{\milestone}{\textbf{\textrm{M}}}
\newcommand{\deliverable}{\textbf{\textrm{D}}}

\lhead{
  \textcolor{gray}{\textit{Matthew Feickert}}\\
  \href{mailto:matthew.feickert@cern.ch}{matthew.feickert@cern.ch}
}
\rhead{\textcolor{gray}{\href{https://urssi.us/blog/2024/11/22/call-for-proposals-for-the-urssi-early-career-fellowship-program/}{URSSI Fellowship Winter 2025}\\%
\thepage/\totalpages{}}}
\chead{\Large\scshape\color[HTML]{CF000F} Project Proposal}

\begin{document}
\pagestyle{fancy}

\section{Project Goals and Objectives}

% * State the track that your project is most closely aligned with.

% * Provide a brief overview of the project including related research, and previous efforts.
% * Provide clear, measurable goals that will be accomplished within 6 months.
% * Include specific objectives that contribute to the broader URSSI mission and research track.

A critical component of research software sustainability is the reproducibility of the software and computing environments software operates and ``lives'' in.
Providing software as a ``package'' --- a standardized distribution of all source or binary components of the software required for use along with identifying metadata --- goes a long way to improved reproducibility of software libraries.
However, more researchers are consumers of libraries than developers of them, but still need reproducible computing environments for research software ``applications'' that may to be run across multiple computing platforms --- e.g. scientific analyses, visualization tools, data transformation pipelines, and artificial intelligence (AI) and machine learning (ML) applications on hardware accelerator platforms (e.g. GPUs).
While workflow engines and Linux containers offer a gold standard for scientific computing reproducibility, they require additional layers of training and software engineering knowledge.
Modern open source multi-platform environment management tools, e.g. \pixi{}~\cite{Arts_pixi}, provide automatic multi-platform hash-level lock file support for all dependencies --- down to the compiler level --- of software on Python or Conda package indexes (e.g. PyPI and conda-forge) while still providing a high level interface well suited for researchers.
We are now at a point where well supported, robust technological solutions exist for even highly complex software applications.
What is currently lacking is the education and training by the broader scientific software community to adopt these technologies and build community standards of practice around them.

\subsection{Project Focus and Goals}

As a \fullprogram{} (\program{}) Fellow, I will develop an open source course (website) on creating reproducible software environments for scientific applications including AI/ML applications, which require specialized hardware accelerator support.
The course will cover using \pixi{} to create and use reproducible software environments for Python applications using software distributed on public package indexes (e.g. PyPI~\cite{PyPI_website} and conda-forge~\cite{conda-forge_community}), but will be scoped broadly to cover applications for \texttt{C}, \texttt{C++}, and Fortran libraries on conda-forge as well.
There will be a particular focus on using these tools for creating environments of AI/ML applications at scale, including all CUDA dependencies, which is becoming a more relevant issue as AI/ML becomes a standard part of modern science.
I will also create a workshop targeting the broader scientific software community to teach the content of the course to researchers from multiple domains.
The workshop will be incubated at the \fullinstitute{} (\institute{}) Data Science Hub (DSH)~\cite{data_science_hub}, and will solicit a national call for applications for workshop participants, whose travel will be supported by the Fellow award funds as covered in the included proposed budget.
A version of the course will also be submitted as a tutorial proposal at the 2025 SciPy conference~\cite{scipy-2025}, and will be recommended as a module for future URSSI workshops.
%
% \textbf{TODO: Mention that pixi is chosen given its features, open source, and that the pixi development team has expressed their support and willingness to collaborate on the proposed project.}

\subsection{Project Objectives and Deliverables}
This project will deliver permissively licensed open source educational material --- intended for community reuse --- and executable examples for real scientific software applications with a focus on AI/ML.
The material will be created with the intent to contribute it as a lesson module to The Carpentries Incubator~\cite{carpentries_incubator_proposals} and, once peer-reviewed, The Carpentries Lab~\cite{carpentries_lab} curriculum, where it will be maintained as an open source community resource.
It will also result in the training of a cohort of early career researchers across multiple scientific domains on best practices for reproducible scientific applications and will additionally identify areas of high complexity and significant difficulty in their current workflows that can either be alleviated or addressed through further technological development.
It will additionally serve as a testing ground for \pixi{} to handle real world complex scientific workflows and identify edge cases that will be addressed upstream in the \pixi{} GitHub repository in collaboration with the \pixi{} developers.

This project proposal is best aligned with the \textbf{Scientific Software Sustainability} track as it provides actionable methods that can be adopted by the broader scientific software community for better software sustainability and reproducibility.
It also has significant overlap with the \textbf{Software Education Research} track, in that project products could be used in future \program{} schools, though the focus is on adoption of technologies and best practices rather than pedagogical research.
%
% \textbf{TODO: Clarify how this is different than MLFlow frozen environments.}

\section{Expected Impact on Scientific Software Community}

Direct and indirect benefits to the research software community.
How will your work improve scientific software development practices?

Reproducible software environments are critical not only for reproducibility of results, but, arguably more importantly, for practically sharing complex scientific environments with many dependencies for researchers to work collaboratively.
The open source website, course, and workshop series together will practically teach researchers how to make their software reusable and their exact software environments easily and immediately shareable.
It is without exaggeration that today researchers can spend hours building bespoke software environments for their work that have complex enough dependency trees that they require great effort to rebuild even on the same machines.
Providing researchers with not only a best practice workflow for lock-file-based reproducible software environments, but also the tools to immediately implement them across multiple platforms will scale the human time savings across science significantly as researchers adopt these methods in their research groups.
It will also increase computational savings by avoiding running of scientific workflows with improperly defined environments that could alter results, fail to reproduce key findings, or result in errors and crashes.
As GPU based workflows are notoriously difficult for sharing their exact environment specifications, this will provide a huge reproducibility benefit.
Additionally, I will share the course widely and encourage other researchers to adopt it in their own research and teaching.

Most scientists do not receive a formal education in software development and engineering and the associated best practices.
However, these skills are vital to tackle the complex scientific questions of today.
Through the course and workshops I will be able to train early career researchers (university and graduate students, and junior postdocs) across multiple scientific domains on how to improve the maintainability of their software, reproducibility of their work, and ultimately the quality of their science.
The impact of investing in the technical education of early career researchers will grow over time as they bring what they have learned to bear in their research and spread their knowledge in new collaborations.
All the course and workshop content, and the website source code, will be open source under a permissive license, making it easy for other researchers to use the course material in their own lectures and workshops.
Access to high quality scientific computing educational resources is an equity issue and open sourcing that knowledge makes it more accessible. The course will also not be static, but will be updated as new improvements and recommendations are made available.
I will encourage sharing of the material and by having the workshops target a wide range of scientific domains I will ensure that this knowledge spreads and helps multiple scientific communities develop better workflows and practices around reproducible and sustainable software development.

\section{Implementation Plan}

Month-by-month breakdown of activities, methods, and approaches.
Include specific milestones and check-in points.

\section{Community Engagement Strategy}

% How will you involve and benefit the broader scientific software community?
% Include specific outreach and collaboration plans.

Through incubating the course at the DSH it will be submitted as a proposed lesson module on sustainable scientific software development for AI/ML applications to The Carpentries Incubator with the goal of having The Carpentries peer-review the course and accept it into The Carpentries Lab curriculum.
The Carpentries curriculum is curated and maintained and taught globally to members of the scientific software community.
The staff at the DSH have experience working with The Carpentries and successfully contributing materials to The Carpentries curriculum~\cite{backhaus_2024_14360351}.
Additionally, the DSH incubated workshops, the national workshop at the DSH, and the SciPy 2025 tutorial will allow for a large number of early career researchers to be trained in sustainable best practices and learn on real scientific problems that they have in their own research programs.

\section{Evaluation Metrics}

% Concrete ways to measure project success.
% Include both quantitative and qualitative metrics.

The project focuses on \textbf{Scientific Software Sustainability} by educating early career researchers on best practices of reproducible scientific analysis, with a focus on AI/ML applications.
Each workshop will have a post-workshop feedback questionnaire that will be used to not only evaluate the clarity of the content and the efficiency at conveying the material, but also the level of information that the participants learned and found \emph{actionably useful} in their own research.

Metrics for success will include:
\begin{itemize}
  \item The total number of participants across all the workshops.
  \item The number of participants who were able to successfully reproducer their own research AI/ML workflows using the information they learned in the workshop.
  \item The acceptance of the course into The Carpentries Incubator.
  \item The acceptance of the course into The Carpentries Lab curriculum.
\end{itemize}

\section{Timeline and Deliverables}

% Specific six-month timeline with monthly milestones and concrete deliverables.
% Example structure:

% \begin{itemize}
%   \item Month 1: Project setup and initial research
%   \item Month 2-3: Development/Investigation phase
%   \item Month 4-5: Testing/Validation/Analysis
%   \item Month 6: Documentation and dissemination
% \end{itemize}

Milestones and deliverables are noted by \milestone{} and \deliverable{}, respectively.

\begin{itemize}
  \item Month 1 (February 2025): Project setup
    \begin{itemize}
      \item Develop a first draft of the course.
      \item \milestone{} Submit tutorial proposal for the course to SciPy 2025.
      \item \milestone{} Begin the process of contributing the course to the The Carpentries by creating a proposal GitHub issue with The Carpentries Incubator~\cite{carpentries_incubator_proposals}.
    \end{itemize}
  \item Month 2 (March 2025): Iteration and revision
    \begin{itemize}
      \item Finalize the first iteration of the course content.
      \item Contract the DSH staff to review and provide feedback on the pedagogical structure of the course.
      \item Contract \texttt{prefix.dev} (the company that develops \texttt{pixi} as an open source community project) engineers for a technical review of the course content.
      \item \milestone{} Create the post-workshop feedback questionnaire in consultation with DSH staff.
    \end{itemize}
  \item Month 3 (April 2025): Incubation and testing
    \begin{itemize}
      \item \deliverable{} Run the first iteration of the workshop incubation at the DSH for \institute{} university students.
      \item \milestone{} Open a call for applications for the national workshop.
    \end{itemize}
  \item Month 4 (May 2025): Incubation and revision
    \begin{itemize}
      \item \deliverable{} Run the second iteration of the workshop incubation at the DSH for \institute{} early career researchers and staff.
      \item \milestone{} Select applicants for acceptance to attend the in-person national workshop.
      \item Revise the course content based on post-workshop feedback questionnaire.
    \end{itemize}
  \item Month 5 (June 2025): Implementation of production course
    \begin{itemize}
      \item \deliverable{} Run the national workshop at the DSH.
      \item Revise the course content based on post-workshop feedback questionnaire.
    \end{itemize}
  \item Month 6 (July 2025): Dissemination with broader community and documentation
    \begin{itemize}
      \item \deliverable{} Teach the course as a tutorial at SciPy 2025.
      \item \deliverable{} Submit the course to be peer-reviewed for approval to be included in The Carpentries Lab~\cite{carpentries_lab}.
      \item \deliverable{} Publication of the anonymized post-workshop feedback questionnaire results as an interactive report generated with MyST Markdown~\cite{executable_books_community}.
    \end{itemize}
\end{itemize}

%
\vspace*{-0.25cm}
\begin{footnotesize}
%
 \bibliographystyle{atlasBibStyleWithTitle}
 \bibliography{main}
%
\end{footnotesize}
\end{document}
